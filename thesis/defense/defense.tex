\documentclass[14pt, aspectratio=1610]{beamer}
\usetheme{CambridgeUS}
\usepackage{xcolor}
\usepackage[newfloat]{minted}
\usemintedstyle{manni}
\usepackage[utf8]{inputenc}
\usepackage[magyar]{babel}
\usepackage{listings}
% \setbeamertemplate{navigation symbols}{}
\setbeamercovered{transparent}

\DeclareUnicodeCharacter{2124}{$\mathbb{Z}$}
\DeclareUnicodeCharacter{2095}{$_h$}
\DeclareUnicodeCharacter{2261}{$\equiv$}

\title{Egész számok definiálása HIT-kkel koherencia szabállyal és Kommutatív gyűrű tulajdonság formalizálása}
\institute{Eötvös Loránd Tudományegyetem}
\author{Balázs Zoltán}
\date{\today}

\setminted{
   frame=lines,
   framesep=2mm,
   baselinestretch=1.2,
   fontsize=\footnotesize,
}

\begin{document}

\maketitle

\begin{frame}[fragile]{Egész szám definíciók}

\begin{itemize}[<+->]
  \item Normálforma (\texttt{zero}, \texttt{suc}, \texttt{pred} - Agda-ban \texttt{pos}, \texttt{negsuc}): pred (succ z) = ?
  \item HIT
    \begin{itemize}[<+->]
      \item Bi-Invertible (\texttt{zero}, \texttt{suc}, \texttt{predr}, \texttt{suc-predr}, \texttt{predl}, \texttt{predl-suc}) \texttt{predl≡predr} kell!
      \item Koherencia szabállyal (\texttt{zero}, \texttt{suc}, \texttt{pred}, \texttt{suc-pred}, \texttt{pred-suc}, \texttt{coh})
    \end{itemize}
\end{itemize}

\end{frame}

\begin{frame}[fragile]{Kommutatív gyűrű}

\begin{itemize}[<+->]
    \item Az egész számok, összeadás és szorzás gyűrűt alkotnak
      \begin{itemize}[<+->]
        \item Az egész számok és összeadás Abel-csoportot alkotnak
          \begin{itemize}[<+->]
            \item Összeadás egységeleme létezik
            \item Összeadás inverz eleme létezik
            \item Összeadás asszociatív
            \item Összeadás kommutatív
          \end{itemize}
        \item Az egész számok és szorzás egységelemes félcsoportot alkotnak
          \begin{itemize}[<+->]
            \item Szorzás egységeleme létezik
            \item Szorzás asszociatív
          \end{itemize}
        \item A szorzás disztributív az összeadás felett
      \end{itemize}
    \item Szorzás kommutatív
\end{itemize}

\end{frame}

\begin{frame}[fragile]{Implementáció menete}

\begin{itemize}[<+->]
  \item Egész számok definiálása
    \begin{onlyenv}<.>
    \begin{minted}{agda}
data ℤₕ : Set where
  zero : ℤₕ
  succ : ℤₕ → ℤₕ
  pred : ℤₕ → ℤₕ
  sec : (z : ℤₕ) → pred (succ z) ≡ z
  ret : (z : ℤₕ) → succ (pred z) ≡ z
  coh : (z : ℤₕ) → congS succ (sec z) ≡ ret (succ z)
    \end{minted}
    \end{onlyenv}
  \item Izomorfizmus a normálforma definícióval
    \begin{onlyenv}<.>
    \begin{minted}{agda}
ℤ-iso : Iso ℤ ℤₕ
ℤ-iso .Iso.fun      = ℤ→ℤₕ
ℤ-iso .Iso.inv      = ℤₕ→ℤ
ℤ-iso .Iso.rightInv = ℤₕ→ℤ→ℤₕ
ℤ-iso .Iso.leftInv  = ℤ→ℤₕ→ℤ
    \end{minted}
    \end{onlyenv}
  \item Iterátor (+eliminátor) és induktív tulajdonság
  \item Műveletek definiálása
  \item Tulajdonságok formalizálása
\end{itemize}

\end{frame}

\begin{frame}{Ez miért új? Mi a haszna?}

\begin{itemize}[<+->]
    \item Az egész számok tényleg kommutatív gyűrűt alkotnak
    \item Egyszerűbb bebizonyítani tulajdonságokat mint a normálforma definícióval
    \item Kézenfoghatóbb HIT definíció mint a Bi-Invertible
    \item Automatikus levezetés összeadást, negálást (kivonást) és szorzást használó egyenlőségekre
    \item Eddig csak elméleti síkon volt belátva
    \item Eliminálhatunk magasabb univerzumba (típusok, Girard paradox, Russel paradox, System U)
\end{itemize}

\end{frame}

\begin{frame}[plain]
  \centering
  \Huge
  Köszönöm a figyelmet!
\end{frame}

\end{document}
