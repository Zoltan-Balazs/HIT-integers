\documentclass[14pt, aspectratio=1610]{beamer}
\usetheme{CambridgeUS}
\usepackage{xcolor}
\usepackage[newfloat]{minted}
\usemintedstyle{manni}
\usepackage[utf8]{inputenc}
\usepackage[magyar]{babel}
\usepackage{listings}
% \setbeamertemplate{navigation symbols}{}
\setbeamercovered{transparent}

\DeclareUnicodeCharacter{2124}{$\mathbb{Z}$}
\DeclareUnicodeCharacter{2095}{$_h$}
\DeclareUnicodeCharacter{2261}{$\equiv$}

\title{HIT egész számok kommutatív gyűrűt alkotnak}
\institute{Eötvös Loránd Tudományegyetem}
\author{Balázs Zoltán}
\date{\today}

\setminted{
   frame=lines,
   framesep=2mm,
   baselinestretch=1.2,
   fontsize=\footnotesize,
}

\begin{document}

\maketitle

\begin{frame}[fragile]{Egész szám definíciók}

\begin{itemize}
    \item<+-> Normálforma (\texttt{zero}, \texttt{suc}, \texttt{pred})
    \item<+-> Jobb normálforma (\texttt{pos}, \texttt{negsuc})
    \item<+-> Bi-Invertible (\texttt{zero}, \texttt{suc}, \texttt{predr}, \texttt{suc-predr}, \texttt{predl}, \texttt{predl-suc}, \texttt{predl≡predr})
    \item<+-> HIT (\texttt{zero}, \texttt{suc}, \texttt{pred}, \texttt{suc-pred}, \texttt{pred-suc}, \texttt{coh})
\end{itemize}

\end{frame}

\begin{frame}[fragile]{Kommutatív gyűrű}

\begin{itemize}
    \item<+-> A halmaz, összeadás és szorzás gyűrűt alkot
      \begin{itemize}
        \item<+-> A halmaz és összeadás Abel-csoportot alkot
          \begin{itemize}
            \item<+-> Összeadás egységeleme létezik
            \item<+-> Összeadás inverz eleme létezik
            \item<+-> Összeadás asszociatív
            \item<+-> Összeadás kommutatív
          \end{itemize}
        \item<+-> A halmaz és szorzás monoid
          \begin{itemize}
            \item<+-> Szorzás egységeleme létezik
            \item<+-> Szorzás asszociatív
          \end{itemize}
        \item<+-> A szorzás disztributív az összeadásra
      \end{itemize}
    \item<+-> Szorzás kommutatív
\end{itemize}

\end{frame}

\begin{frame}[fragile]{Implementáció menete}

\begin{minted}{agda}
data ℤₕ : Set where
  zero : ℤₕ
  succ : ℤₕ → ℤₕ
  pred : ℤₕ → ℤₕ
  sec : (z : ℤₕ) → pred (succ z) ≡ z
  ret : (z : ℤₕ) → succ (pred z) ≡ z
  coh : (z : ℤₕ) → congS succ (sec z) ≡ ret (succ z)
\end{minted}

\begin{itemize}
    \item<+-> Halmaz definiálása
    \item<+-> Izomorfizmus a normálforma definícióval
    \item<+-> Iterátor (+eliminátor) és induktív tulajdonság
    \item<+-> Műveletek (és tulajdonságok) Agda-ban való definiálása (és formalizálása)
\end{itemize}

\end{frame}

\begin{frame}{Ez miért új? Mi a haszna?}

\begin{itemize}
    \item<+-> Eddig csak elméleti síkon volt belátva
    \item<+-> Eliminálhatunk magasabb dimenziókra (halmazok felé, pl. groupoidokra)
    \item<+-> Szebb definíció mint a Bi-Invertible egész számok
    \item<+-> Az egész számok tényleg kommutatív gyűrűt alkotnak
    \item<+-> Automatikus levezetés összeadást, negálást (kivonást) és szorzást használó egyenlőségekre
\end{itemize}

\end{frame}

\begin{frame}[plain]
  \centering
  \Huge
  Köszönöm a figyelmet!
\end{frame}

\end{document}
