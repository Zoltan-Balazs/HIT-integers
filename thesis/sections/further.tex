\chapter{Further Work}
\label{ch:further}

Unfortunately not everything described in the topic announcer has been achieved, in this chapter we will discuss these.

\section{Proving the Coherent case}
While significant work has been done to prove that converting a 'coh' HIT integer to the standard type and back to HIT is the exact same value, this case isn't done. Intuitively just like for the 1-dimensional constructors 'sec' and 'ret' we were able to prove this equality, this should hold true for the 2-dimensional constructor of 'coh' as well.

The problem lies in the fact that the boundaries required by a 2-dimensional constructor create a 3-dimensional boundary cube for us. To prove this, we would have to draw a 4-dimensional cube (tesseract), for this, we would have to draw the 8 cells (3D cubes) that make up this tesseract to even have a chance at satisfying these boundaries. From any point on the wanted boundary cube, there is a given proof to move along the 4th dimension and back to another point on the wanted boundary cube.

\[\begin{tikzcd}
	& H && G \\
	D && C \\
	& E && F & j & i \\
	A && B && \bullet & k
	\arrow[no head, from=1-2, to=1-4]
	\arrow[no head, from=2-1, to=1-2]
	\arrow[no head, from=2-1, to=2-3]
	\arrow[no head, from=2-3, to=1-4]
	\arrow[dashed, no head, from=3-2, to=1-2]
	\arrow[dashed, no head, from=3-2, to=3-4]
	\arrow[no head, from=3-4, to=1-4]
	\arrow[no head, from=4-1, to=2-1]
	\arrow[dashed, no head, from=4-1, to=3-2]
	\arrow[no head, from=4-1, to=4-3]
	\arrow[no head, from=4-3, to=2-3]
	\arrow[no head, from=4-3, to=3-4]
	\arrow[from=4-5, to=3-5]
	\arrow[from=4-5, to=3-6]
	\arrow[from=4-5, to=4-6]
\end{tikzcd}\]

Where, interestingly enough, 4-4 pair of point are equal as such:
\begin{compactitem}
  \item A = E = ℤ→ℤₕ (sucℤ (predℤ (sucℤ (ℤₕ→ℤ z))))
  \item B = F = succ (pred (succ z))
  \item C = G = succ z
  \item D = H = ℤ→ℤₕ (sucℤ (ℤₕ→ℤ z))
\end{compactitem}

While one could say that this reduces our cube to a square, unfortunately the situation is different, as we also have the face boundaries required for our proof (here we unrolled the cube to its 6 squares to make reading easier):

\[\begin{tikzcd}
	&& D && H \\
	&&& {k_0} \\
	D && A && E && H \\
	& {i_0} && {j_0} && {i_1} \\
	C && B && F && G \\
	&&& {k_1} \\
	&& C && G \\
	&&& {j_1} \\
	&& D && H
	\arrow[no head, from=1-3, to=1-5]
	\arrow[no head, from=1-3, to=3-3]
y	\arrow[no head, from=1-5, to=3-5]
	\arrow[no head, from=3-1, to=5-1]
	\arrow[no head, from=3-3, to=3-1]
	\arrow[no head, from=3-3, to=3-5]
	\arrow[no head, from=3-3, to=5-3]
	\arrow[no head, from=3-5, to=3-7]
	\arrow[no head, from=3-5, to=5-5]
	\arrow[no head, from=5-1, to=5-3]
	\arrow[no head, from=5-3, to=5-5]
	\arrow[no head, from=5-3, to=7-3]
	\arrow[no head, from=5-5, to=5-7]
	\arrow[no head, from=5-7, to=3-7]
	\arrow[no head, from=7-3, to=7-5]
	\arrow[no head, from=7-5, to=5-5]
	\arrow[no head, from=7-5, to=9-5]
	\arrow[no head, from=9-3, to=7-3]
	\arrow[no head, from=9-5, to=9-3]
\end{tikzcd}\]

Here, our required boundaries will be:
\begin{compactitem}
  \item $k_0$ = \mintinline{agda}{ℤ→ℤₕ (cohℤ (ℤₕ→ℤ z) i j)}
  \item $k_1$ = \mintinline{agda}{coh z i j}
  \item $j_0$ = \mintinline{agda}{ℤₕ→ℤ→ℤₕ (coh z i i0) k}
  \item $j_1$ = \begin{minted}[linenos=false]{agda}
         hcomp
         (doubleComp-faces (λ _ → ℤ→ℤₕ (sucℤ (ℤₕ→ℤ z)))
          (λ i₁ → succ (ℤₕ→ℤ→ℤₕ z i₁)) k)
         (ℤ→ℤₕ-sucℤ (ℤₕ→ℤ z) k)
         \end{minted}
  \item $i_0$ = \mintinline{agda}{ℤₕ→ℤ→ℤₕ (coh z i0 j) k}
  \item $i_1$ = \mintinline{agda}{ℤₕ→ℤ→ℤₕ (coh z i1 j) k}
\end{compactitem}

We have to find out what the 7 other cubes required for the 4D tessaract are. After this, if we correctly define the the helper function's type (for which we will more than likely need additional helper functions as well, for example to prove that converting a standard integer that has 'sucℤ' applied, while already having 'predℤ' and 'sucℤ' applied is the same as converting a standard integer, then applying 'succ', 'pred' and 'succ' - the 'succ (pred (succ z))' part of the HIT Integer's 'coh' rule) we still aren't finished, as we need to pattern match on this value to successfully define the function. As seen from the 'predSuc' and 'sucPred' helper functions, this will more than likely mean 3 (or even 4) cases, where we will need to fill in 7 values each.
Proving the discrete property might've been better after all.

\section{Relations}
With an equivalence relation defined (which is default for all types), we can, for example, define the injection property for the 'succ' and 'pred' constructors as such:
\begin{minted}{agda}
succ-inj : ∀ m n → succ m ≡ succ n → m ≡ n
succ-inj m n eq = sym (sec m) ∙ congS pred eq ∙ sec n

pred-inj : ∀ m n → pred m ≡ pred n → m ≡ n
pred-inj m n eq = sym (ret m) ∙ congS succ eq ∙ ret n
\end{minted}

\section{Division (and Modulus)}
It is possible to write a division operator for integers. One might ask how that is possible, since, for example, 3 / 2 results in something that is not an integer. The answer lies in rounding down (or up) the result. Unfortunately, due to this, we cannot use our iterator to define this operator, as it is non-reversible, we need to pattern match.
This brings its own problems, as for our 1- and 2-dimensional constructors ('sec', 'ret' and 'coh') it turns out to be a laborous task.
There is also the problem of dividing by zero, for which the standard Agda library introduces an interesting solution. Introducing a 'NonZero' record, which can ensure that the divisor is not zero.
For the operation itself, we introduce a helper function, which will handle the cases. A simple 'succ' case division is introduced. (As it is seen in the Agda standard library\cite{div-nat})
These same principles also apply for the modulus operator:
\begin{minted}{agda}
record NonZero (z : ℤₕ) : Set where
  field
    nonZero : Bool→Type (not (z ≡ zero))

infixl 7 _/_ _%_

div-helper : (k m n j : ℤₕ) → ℤₕ
div-helper k m zero     j        = k
div-helper k m (succ n) zero     = div-helper (succ k) m n m
div-helper k m (succ n) (succ j) = div-helper k        m n j

_/_ : (dividend divisor : ℤₕ) . {{_ : NonZero divisor}} → ℤₕ
m / (succ n) = div-helper zero n m n

mod-helper : (k m n j : ℤₕ) → ℤₕ
mod-helper k m zero      j       = k
mod-helper k m (succ n)  zero    = mod-helper zero     m n m
mod-helper k m (succ n) (succ j) = mod-helper (succ k) m n j

_%_ : (dividend divisor : ℤₕ) . {{_ : NonZero divisor}} → ℤₕ
m % (succ n) = mod-helper zero n m n
\end{minted}

\section{Exponentiation}
Exponentiation is an interesting topic when talking about integers. First of all, our type definition seems out of place, as our second parameter is not an integer, but a natural number. From elemenary school, we know that raising an integer to a negative power is equivalent to 1 divided by the integer to the absolute value of the power. This number is usually not an integer, so we need to strictly ensure that our exponent is a non-negative number.
This restriction (along with our non-exhaustive division operation resulting in us not able to give a correct reversible operation to multiplication) results in us not being able to use our iterator to define this operation, we need to pattern match once again.
As usual, defining the results of our 1- and 2-dimensional constructors ('sec', 'ret' and 'coh') is quite laborous. For this reason, we will only define the 0-dimensional constructors and our base element:
\begin{minted}{agda}
infixr 8 _^^_

_^^_ : ℤₕ → ℕ → ℤₕ
a          ^^ zero  = succ zero
zero       ^^ suc b = zero
x@(succ a) ^^ suc b = x * (x ^^ b)
x@(pred a) ^^ suc b = x * (x ^^ b)
\end{minted}
With this operation defined, we could prove properties of exponentiation, for example: products of powers: $a^m \cdot a^n = a^{m+n}$, power of a power: $(a^m)^n = a^{m^n}$, power of a product: $(a \cdot b)^m = a^m \cdot b^m$.
Other properties of exponentiation require division, which, as we described, is non reversible.
Unfortunately, since our second parameter is not an integer, we cannot use our induction property to prove the before-mentioned properties, we would need to pattern match, which, once again, would introduce the need to prove the cases for our 1- and 2-dimensional constructors, for which we haven't defined what the exponentiation operation is, so it would be an impossible task.
