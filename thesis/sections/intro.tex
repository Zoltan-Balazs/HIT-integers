\chapter{Introduction}
\label{ch:intro}

\section{Type Definition}

In Homotopy Type Theory, we can define integers in numerous ways (one of them is using bi-invertible maps, which is a slightly less elegant version than ours\cite{10.1145/3373718.3394760}, since it involves 2 'pred' constructors, and ditches the 'coh' rule, in favour of being less complicated to prove properties. This bi-invertible map version is also included in the cubical Agda library\cite{biinv-agda}, which will prove useful for us), all with their own pros and cons. Following Paolo Capriotti's idea (presented by Thorsten Altenkirch\cite{bonn_2018}), our higher inductive type definition will be the following:
\begin{minted}{agda}
data ℤₕ : Set₀ where
  zero : ℤₕ
  succ : ℤₕ → ℤₕ
  pred : ℤₕ → ℤₕ
  sec : (z : ℤₕ) → pred (succ z) ≡ z
  ret : (z : ℤₕ) → succ (pred z) ≡ z
  coh : (z : ℤₕ) → congS succ (sec z) ≡ ret (succ z)
\end{minted}
With this definition, we have the base element 'zero', as well as 'succ' and 'pred' as constructors, to increment and decrement the integer value respectively. We then postulate that they are inverse to each other with the inclusion of 'sec' ('section', often seen in the Agda cubical library as 'predSuc') and 'ret' ('retraction', often seen as 'sucPred' in the same library). With a 'coh' ('coherence') constructor, we define an equivalence of equivalences, this constructor will introduce most of the challenges when trying to work with our integer definition. With the inclusion of this last condition, we also say that succ is a half-adjoint equivalence, something that we can use to our advantage in cubical Agda:
\begin{minted}{agda}
isHAℤₕ : isHAEquiv succ
isHAℤₕ .isHAEquiv.g    = pred
isHAℤₕ .isHAEquiv.linv = sec
isHAℤₕ .isHAEquiv.rinv = ret
isHAℤₕ .isHAEquiv.com  = coh
\end{minted}
Using this, we can define a sort of inverse coherence rule, a coherence that inverses the equivalence by applying 'pred' to 'ret', and checking that it is equal to passing the 'pred' value to 'sec':
\begin{minted}{agda}
hoc : (z : ℤₕ) → congS pred (ret z) ≡ sec (pred z)
hoc = com-op isHAℤₕ
\end{minted}
'com-op' simply uses the previously given fields and does the work for us, if our type is right, by correctly combining paths with 'hcomp' to match the wanted boundary.
This rule will be useful later on, when defining operations on our integer type. (Specifically when defining negation)

\section{Why Cubical Agda}

Since established that we want to use a Higher Inductive Type (HIT), which is not available in Martin-Löf Type Theory (MLTT). Agda, by default, doesn't support HITs as it is based on MLTT; an extension of it does however support CTT. While HoTT can essentially be summarized\cite{hottbook} by being MLTT with an additional Univalence axiom (where univalence states the following: (A ≃ B) ≃ (A = B), where A and B are types). Cubical Agda doesn't directly support HoTT; instead, it is a closely related version of Type Theory called Cubical Type Theory (CTT). CTT instead can be summarized as MLTT, with the interval type and its rules, the transport function and its rules, the glue type and hcomp. In CTT, univalence will not be an axiom but something that can be proven\cite{cohen2016cubicaltypetheoryconstructive}.

Since we need to use a proof assistant that supports HITs, which CTT supports,\cite{coquand2018higherinductivetypescubical}, we are forced to use Cubical Agda\cite{10.1145/3341691}, as there are no other proof assistants that do so. This will definitionally bring the added work that we will be forced to satisfy the boundaries required, as of the writing of this paper, there is no automatic way for these to be filled, but that might be subject to change\cite{doré2024automatingboundaryfillingcubical}.

The main benefit of our HIT integers will be that we can eliminate to higher dimensions (more specifically, above set, for example, to groupoid). The same could be said about Bi-Invertible integers, which use 2 'pred' constructors instead, and ditch the 'coherent' rule, however, a single 'pred' constructor with a 'coherent' rule is more elegant. There is also a definition of integers with the added 'isSet' constructor, while you could also eliminate with this version, you couldn't eliminate above set. Moreover, when defining your indunction property, you would also have to define the case for the 'isSet' constructor.

\section{Commutative Ring}
Our question is the following: Is this definition of integers a correct one, is it a set with decidable equality, and if so, do they form a commutative ring? (While this should be fairly obvious, integers do form a commutative ring, with the higher inductive type definition of integers, this hasn't been formally proven yet.) Moreover, what does it mean for integers to form a commutative ring? We will have to prove the following:

\begin{compactitem}
  \item The set and two binary operations (here: addition and multiplication) form a ring:
  \begin{compactitem}
    \item The set and addition form an abelian group:
    \begin{compactitem}
    \item Addition is associative
    \item The identity element exists
    \item An inverse element exists
    \item Addition is commutative
    \end{compactitem}
  \item The set is monoid under multiplication:
    \begin{compactitem}
    \item Multiplication is associative
    \item The identity element exists
    \end{compactitem}
  \item Multiplication distributes over addition
  \end{compactitem}
  \item Multiplication is commutative
\end{compactitem}
Before we prove these, we will dive into proving some other useful properties first.
