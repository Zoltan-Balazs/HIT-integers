\chapter{Monoid (Multiplication)}
\label{ch:monoid}

\section{Multiplication Operation}
To define the multiplication operation, we will once again use our iterator. In this case however, we will have to do a bit of extra work. We cannot provide an Isomorphism in such an easy way as we did with addition. From elementary grade mathmatics, we know that multiplication is just repeated addition, for example if we want to calculate '3 * 4', we would do '4 + 4 + 4', or adding together '4' (right element) '3' (left element) times. So our main idea is to start with zero, and repeat the addition of the right parameter exactly the number of left parameter times (if the left parameter is negative, we will just add the inverted right parameter the absolute value of left parameter times).

As before, our function of having this desired effect will be defining the 'function' of isomorphism as 'z +', conversely, we can define the 'inverse function' as adding the negated value of this 'z': '- z +' since this will invert the effect of adding z to a number.

Next, we will need to prove the 'retraction' property of this.
\begin{theorem}
  Adding a negated value to a number, then the non-negated value of the same value, we get back the original number: ∀ m, n ∈ ℤ: n + ((- n) + m) = m
\end{theorem}

\begin{proof}
  To prove this, we will first use the associative property of addition to match the left-hand side. Supplying 'n', '- n' and 'm' results in the following equation: n + ((- n) + m) ≡ n + (- n) + m. We use transitivity to change the right-hand side of the equation. Getting under the '+ m' part, we can apply the existence of the right inverse element of addition with 'n' parameter. This will replace 'n + (- n)' with 0: n + ((- n) + m) ≡ 0 + m. We know that definitionally 0 + m = m, this property is proven.
\end{proof}

Afterwards, we will have to prove the 'section' property of this.
\begin{theorem}
  Adding a non-negated value to a number, then the negated value of the same value, we get back the original number: ∀ m, n ∈ ℤ: (- n) + (n + m) = m
\end{theorem}

\begin{proof}
  Similarly to the previous proof, to prove this, we will first use the associative property of addition to match the left-hand side. Supplying '- n', 'n' and 'm' results in the following equation: (- n) + (n + m) ≡ (- n) + n + m. We use transitivity to change the right-hand side of the equation. Getting under the '+ m' part, we can apply the existence of the left inverse element of addition with 'n' parameter. This will replace '(- n) + n' with 0: (- n) + (n + m) ≡ 0 + m. We know that definitionally 0 + m = m, this property is also proven.
\end{proof}

Next, we can convert this isomorphism to the proof that 'z +' is an equivalence ('isEquiv'), since we will need to supply parameters this time (instead of placing the constructors), we cannot directly convert to the equivalence ('Equiv'). Afterwards we will define our ordered pair of the equivalence, with a parameter this time.

Lastly, when defining our operation, we will use our previously mentioned idea of placing 'n +' (or '- n +') 'm' times on zero.

\begin{minted}{agda}
Iso-n+-ℤₕ : (z : ℤₕ) → Iso ℤₕ ℤₕ
Iso.fun      (Iso-n+-ℤₕ z)   = z +_
Iso.inv      (Iso-n+-ℤₕ z)   = - z +_
Iso.rightInv (Iso-n+-ℤₕ n) m = +-assoc n (- n) m ∙ cong (_+ m) (+-invʳ n)
Iso.leftInv  (Iso-n+-ℤₕ n) m = +-assoc (- n) n m ∙ cong (_+ m) (+-invˡ n)

isEquiv-n+-ℤₕ : ∀ z → isEquiv (z +_)
isEquiv-n+-ℤₕ z = isoToIsEquiv (Iso-n+-ℤₕ z)

Equiv-n+-ℤₕ : (z : ℤₕ) → ℤₕ ≃ ℤₕ
Equiv-n+-ℤₕ z = z +_ , isEquiv-n+-ℤₕ z

_*_ : ℤₕ → ℤₕ → ℤₕ
m * n = ℤₕ-ite zero (Equiv-n+-ℤₕ n) m
\end{minted}

Definitionally, the following hold true for the multiplication operation in Agda:
\begin{minted}{agda}
zero   * n ≡ zero
succ m * n ≡ n + m * n
pred m * n ≡ (- n) + m * n
\end{minted}
Note that the symmetric version of these also holds true.

We could once again try to define multiplication by pattern matching. In this case however, even defining 'sec' is tricky.
% !!BOUNDARY!!
\begin{minted}{agda}
_*_ : ℤₕ → ℤₕ → ℤₕ
zero      * b = zero
succ a    * b = a * b + b
pred a    * b = a * b - b
sec a i   * b = ?
ret a i   * b = ?
coh a i j * b = ?
\end{minted}

\begin{listing}[H]
\section{Identity Element}

\begin{theorem}
  Left identity element exists for multiplication: ∃ id ∈ ℤ, ∀ z ∈ ℤ: id * z = z
\end{theorem}

\begin{proof}
  In Agda, we will rewrite the statement as such: ∀ z ∈ ℤ → 1 * z = z, since we know that the identity element is 1.

  1 * z = z + 0, so we will reuse our theorem of proving that a right identity element exists for addition.
\end{proof}

\begin{minted}{agda}
*-idˡ : ∀ z → succ zero * z ≡ z
*-idˡ = +-idʳ
\end{minted}
\caption{Agda proof of multiplication having a left identity element}
\end{listing}

\begin{theorem}
  Right identity element exists for multiplication: ∃ id ∈ ℤ, ∀ z ∈ ℤ: z * id = z
\end{theorem}

\begin{proof}
  As in the case of the left identity element, we will rewrite this statement in Agda as such: ∀ z ∈ ℤ → z * 1 = z, since our right identity element is also going to be 1.

  For this, we won't use our induction property. Instead, we will rewrite the left-hand side of the equation to 1 * z, using the commutative property of multiplication (which we prove later on). Now we will need to prove that 1 * z = z, which is the proof of the left identity element existing.
\end{proof}

\begin{listing}[H]
\begin{minted}{agda}
*-idʳ : ∀ z → z * succ zero ≡ z
*-idʳ z = *-comm z (succ zero) ∙ *-idˡ z
\end{minted}
\caption{Agda proof of multiplication having a right identity element}
\end{listing}

\section{Multiplication Distributes over Addition}
As was the case for the identity element and inverse element proofs, we will have to prove both the left and right side for distributivity.

\begin{theorem}
  Multiplication is distributive over addition on the left side: ∀ m, n, o ∈ ℤ: (m + n) * o = (m * o) + (n * o)
\end{theorem}

\begin{proof}
  Using our induction property:

  For the base case (m = 0) : (0 + n) * o = (0 * o) + (n * o) is a reflection.

  For the succ case: ((succ m) + n) * o = ((succ m) * o) + (n * o). Definitionally, we know that (succ m) * o = o + m * o, as such, our equation is essentially the same as: o + (m * n) * o = o + (m * o) + (n * o). For the left-hand side of the equation, we will note that disregarding the beginning o + part, the equation is the same as the original equality's left-hand side, using congruence, we can substitute on the left-hand side. Using this, our right-hand side will look like this: o + (m * o + n * o), this is almost the exact thing that is on the right-hand side, apart from the fact that addition is paranthesized from the left (our original o + (m * n) + (n * o) = (o + (m * n)) + (n * o)), using the associative property of addition, we can rearrange the parts to be correctly parenthesized.

  For the pred case: ((pred m) + n) * o = ((pred m) * o) + (n * o). Once again, definitionally we know that (pred m) * o = (- o) + m * o. We can rewrite the equation once again as: - o + (m * n) * o = - o + (m * o) + (n * o). We can use the exact same steps as in the succ case (Using transitivity, rewriting the left-hand side using congruence and the original equality, rewriting the right hand side with the associative property of addition), noting the difference that for congruence and the first part of the associative property of addition, we will have to provide '- o' as a parameter instead of 'o'.
\end{proof}

\begin{listing}[H]
\begin{minted}{agda}
*-distribˡ-+ : ∀ m n o → (m + n) * o ≡ (m * o) + (n * o)
*-distribˡ-+ = ℤₕ-ind-prop
  (λ _ → isPropΠ2 λ _ _ → isSetℤₕ _ _)
  (λ n o → refl)
  (λ m p n o → cong (o +_) (p n o) ∙ +-assoc o (m * o) (n * o))
  (λ m p n o → cong (- o +_) (p n o) ∙ +-assoc (- o) (m * o) (n * o))
\end{minted}
\caption{Agda proof of multiplication being left distributive to addition}
\end{listing}

\begin{theorem}
  Multiplication is distributive over addition on the right side: ∀ m, n, o ∈ ℤ: m * (n + o) = (m * n) + (m * o)
\end{theorem}

\begin{proof}
  Using the commutative property of multiplication, we can rewrite m * (n + o) to (n + o) * m, this equation looks the same as the left-hand side our previous theorem (multiplication is distributive over addition on the left side). Applying it, with the correct parameters (n, o, m as 'm', 'n' and 'o' respectively), we get the equation of n * m + o * m. These are the exact parts that we need, we will just need to use the commutative property of multiplication twice to switch around the terms. (Note: we can use the congruences in any way, i.e. it doesn't matter if the rearrange the left or the right part first.)
\end{proof}

\begin{listing}[H]
\begin{minted}{agda}
*-distribʳ-+ : ∀ m n o → m * (n + o) ≡ (m * n) + (m * o)
*-distribʳ-+ m n o = *-comm m (n + o) ∙ *-distribˡ-+ n o m ∙ cong (n * m +_) (*-comm o m) ∙ cong (_+ m * o) (*-comm n m)
\end{minted}
\caption{Agda proof of multiplication being right distributive to addition}
\end{listing}

\section{Associative}
\begin{minted}[fontsize=\small]{agda}
inv-hom-ℤₕ : ∀ m n → - (m + n) ≡ (- m) + (- n)
inv-hom-ℤₕ = ℤₕ-ind-prop
  (λ _ → isPropΠ λ _ → isSetℤₕ _ _)
  (λ n → refl)
  (λ m p n → cong pred (p n))
  (λ m p n → cong succ (p n))

*-inv : ∀ m n → m * (- n) ≡ - (m * n)
*-inv = ℤₕ-ind-prop
  (λ _ → isPropΠ λ _ → isSetℤₕ _ _)
  (λ n → refl)
  (λ m p n → cong (- n +_) (p n) ∙ sym (inv-hom-ℤₕ n (m * n)))
  (λ m p n → cong (- (- n) +_) (p n) ∙ sym (inv-hom-ℤₕ (- n) (m * n)))

inv-* : ∀ m n → (- m) * n ≡ - (m * n)
inv-* m n = *-comm (- m) n ∙ *-inv n m ∙ cong (-_) (*-comm n m)

*-assoc : ∀ m n o → m * (n * o) ≡ (m * n) * o
*-assoc = ℤₕ-ind-prop
  (λ _ → isPropΠ2 λ _ _ → isSetℤₕ _ _)
  (λ n o → refl)
  (λ m p n o → cong (n * o +_) (p n o) ∙ *-distribʳ-+ n (m * n) o)
  (λ m p n o → cong (- (n * o) +_) (p n o) ∙ cong (_+ m * n * o) (sym (inv-* n o)) ∙ *-distribʳ-+ (- n) (m * n) o)
\end{minted}
